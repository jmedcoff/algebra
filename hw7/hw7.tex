\documentclass{amsart}
\usepackage{amsmath}
\usepackage{amssymb}
\usepackage{geometry}
\geometry{a4paper}

\newcommand{\problem}[1]{\noindent{\textbf{#1}}}

\title{Homework 7}
\author{Jason Medcoff}
\date{}

% 3.8 (page 303):  97, 100, 103

% 4.1 (page 341): 2, 7, 8, 11, 15, 18

% 4.1 (page 353): 24, 27, 28

\begin{document}
	\maketitle
	
	\problem{93.}
	% iso thm for rings
	We can use the isomorphism theorem by showing that there is a homomorphism $f: R[x]/ \to R$, demonstrating the kernel of $f$ to be $(x)$, and showing that the image of $f$ is $R$ itself. So first, choose $f$ such that $r(x)=r_0+\ldots+r_n x^n \mapsto r_0$; in other words, $f$ is the evaluation function that sends polynomials $r(x)$ to $r(0)$.
	
	This map is well defined since equality of polynomials is defined by equality of coefficients. The map is a homomorphism since
	$$ f(r(x) + s(x)) = r_0 + s_0 = f(r(x)) + f(s(x)) $$
	and
	$$ f(r(x)s(x)) = r_0s_0 = f(r_0)f(s_0), $$
	and furthermore, the zero polynomial obviously maps to zero in $R$. Thus $f$ is a homomorphism.
	
	The kernel of $f$ is given by
	\begin{equation*}
	\begin{split}
	\ker f &= \{ r(x) \in R[x] : f(r(x)) = 0 \} \\
	&= \{ r_1 x + \ldots + r_n x^n \} \\
	&= (x)
	\end{split}
	\end{equation*}
	since polynomials without constant terms are divisible by $x$ without remainder. Finally, the image of $f$ is clearly the entirety of $R$ since one can choose any constant polynomial $r(x) = r_0$ with $r_0 \in R$. Thus by the first isomorphism theorem, there exists an isomorphism between $R[x]/(x)$ and $R$.
	
	$\newline$
	\problem{97.}
	i) $F$ is well defined since for two conjugacy classes $[a] = [b]$,
	$$ F([a]) = a^p \mod q 
	
	
	
	
	
	
	
	
	
	
	
	
	
\end{document}