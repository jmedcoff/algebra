\documentclass{amsart}
\usepackage{amsmath}
\usepackage{amssymb}
\usepackage{geometry}
\geometry{a4paper}

\newcommand{\problem}[1]{\noindent{\textbf{#1}}}

\title{Homework 7}
\author{Jason Medcoff}
\date{}

% 3.8 (page 303):  97, 100, 103

% 4.1 (page 341): 2, 7, 8, 11, 15, 18

% 4.1 (page 353): 24, 27, 28

\begin{document}
	\maketitle
	
	\problem{93.}
	% iso thm for rings
	We can use the isomorphism theorem by showing that there is a homomorphism $f: R[x]/ \to R$, demonstrating the kernel of $f$ to be $(x)$, and showing that the image of $f$ is $R$ itself. So first, choose $f$ such that $r(x)=r_0+\ldots+r_n x^n \mapsto r_0$; in other words, $f$ is the evaluation function that sends polynomials $r(x)$ to $r(0)$.
	
	This map is well defined since equality of polynomials is defined by equality of coefficients. The map is a homomorphism since
	$$ f(r(x) + s(x)) = r_0 + s_0 = f(r(x)) + f(s(x)) $$
	and
	$$ f(r(x)s(x)) = r_0s_0 = f(r_0)f(s_0), $$
	and furthermore, the zero polynomial obviously maps to zero in $R$. Thus $f$ is a homomorphism.
	
	The kernel of $f$ is given by
	\begin{equation*}
	\begin{split}
	\ker f &= \{ r(x) \in R[x] : f(r(x)) = 0 \} \\
	&= \{ r_1 x + \ldots + r_n x^n \} \\
	&= (x)
	\end{split}
	\end{equation*}
	since polynomials without constant terms are divisible by $x$ without remainder. Finally, the image of $f$ is clearly the entirety of $R$ since one can choose any constant polynomial $r(x) = r_0$ with $r_0 \in R$. Thus by the first isomorphism theorem, there exists an isomorphism between $R[x]/(x)$ and $R$.
	
	$\newline$
	\problem{97.}
	
	\textbf{Lemma.} In a finite field of prime order $n$, $(a+b)^n = a^n + b^n$. By binomial coefficients,
	$$ (a+b)^n = \sum_{k=0}^{n} {n \choose k} a^{n-k} b^k $$ 
	where $$ {n \choose k} = \frac{n!}{k!(n-k)!} $$
	but for $n>k$, $n$ divides $n!$ but not $k!$. Then the coefficients of all terms but the first and the last are divisible by the characteristic. All that is left is $a^n + b^n$.
	
	i) $F$ obeys the additive homomorphism rule since
	$$ F(a+b) = a^p + b^p = (a+b)^p = F(a) + F(b) $$
	since by binomial coefficients,
	$$ (a+b)^p = \sum_{k=0}^{p} {p \choose k} a^{p-k} b^k $$ 
	where $$ {p \choose k} = \frac{p!}{k!(p-k)!} $$
	but since 
	
	$\newline$
	\problem{100.}
	i) By the lemma in 97, we can write $x^4 + 1 = x^4 + 1^4 = (x+1)^4$.
	
	ii) We can write
	\begin{equation*}
	\begin{split}
	(x^2+ax+b)(x^2+cx+d) &= x^4 + cx^3 + dx^2 + ax^3 + acx^2 + adx + bx^2 + bcx + bd \\
	&= x^4 + (c+a)x^3 + (d+ac+b)x^2 + (ad+bc)x + (bd) \\
	&= x^4 + 1
	\end{split}
	\end{equation*}
	and by equating coefficients, clearly $bd=1$, $c+a=0$ or $c=-a$, $d+ac+b = 0$, and $ad+bc = 0$. The latter two can be written as $d+b-a^2=0$ and $ad-ab=0$ or $a(d-b)=0$.
	
	iii) Suppose $b^2 \equiv -1 \mod p$. 
	
	
	
	
	
	
	
	
	
	
	
\end{document}